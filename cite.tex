\documentclass[]{article}
\usepackage{lmodern}
\usepackage{amssymb,amsmath}
\usepackage{ifxetex,ifluatex}
\usepackage{fixltx2e} % provides \textsubscript
\ifnum 0\ifxetex 1\fi\ifluatex 1\fi=0 % if pdftex
  \usepackage[T1]{fontenc}
  \usepackage[utf8]{inputenc}
\else % if luatex or xelatex
  \ifxetex
    \usepackage{mathspec}
    \usepackage{xltxtra,xunicode}
  \else
    \usepackage{fontspec}
  \fi
  \defaultfontfeatures{Mapping=tex-text,Scale=MatchLowercase}
  \newcommand{\euro}{€}
\fi
% use upquote if available, for straight quotes in verbatim environments
\IfFileExists{upquote.sty}{\usepackage{upquote}}{}
% use microtype if available
\IfFileExists{microtype.sty}{%
\usepackage{microtype}
\UseMicrotypeSet[protrusion]{basicmath} % disable protrusion for tt fonts
}{}
\usepackage[margin=1in]{geometry}
\ifxetex
  \usepackage[setpagesize=false, % page size defined by xetex
              unicode=false, % unicode breaks when used with xetex
              xetex]{hyperref}
\else
  \usepackage[unicode=true]{hyperref}
\fi
\hypersetup{breaklinks=true,
            bookmarks=true,
            pdfauthor={Carl Boettiger},
            pdftitle={An introduction to knitcitations},
            colorlinks=true,
            citecolor=blue,
            urlcolor=blue,
            linkcolor=magenta,
            pdfborder={0 0 0}}
\urlstyle{same}  % don't use monospace font for urls
\usepackage{natbib}
\bibliographystyle{plainnat}
\usepackage{color}
\usepackage{fancyvrb}
\newcommand{\VerbBar}{|}
\newcommand{\VERB}{\Verb[commandchars=\\\{\}]}
\DefineVerbatimEnvironment{Highlighting}{Verbatim}{commandchars=\\\{\}}
% Add ',fontsize=\small' for more characters per line
\usepackage{framed}
\definecolor{shadecolor}{RGB}{248,248,248}
\newenvironment{Shaded}{\begin{snugshade}}{\end{snugshade}}
\newcommand{\KeywordTok}[1]{\textcolor[rgb]{0.13,0.29,0.53}{\textbf{{#1}}}}
\newcommand{\DataTypeTok}[1]{\textcolor[rgb]{0.13,0.29,0.53}{{#1}}}
\newcommand{\DecValTok}[1]{\textcolor[rgb]{0.00,0.00,0.81}{{#1}}}
\newcommand{\BaseNTok}[1]{\textcolor[rgb]{0.00,0.00,0.81}{{#1}}}
\newcommand{\FloatTok}[1]{\textcolor[rgb]{0.00,0.00,0.81}{{#1}}}
\newcommand{\ConstantTok}[1]{\textcolor[rgb]{0.00,0.00,0.00}{{#1}}}
\newcommand{\CharTok}[1]{\textcolor[rgb]{0.31,0.60,0.02}{{#1}}}
\newcommand{\SpecialCharTok}[1]{\textcolor[rgb]{0.00,0.00,0.00}{{#1}}}
\newcommand{\StringTok}[1]{\textcolor[rgb]{0.31,0.60,0.02}{{#1}}}
\newcommand{\VerbatimStringTok}[1]{\textcolor[rgb]{0.31,0.60,0.02}{{#1}}}
\newcommand{\SpecialStringTok}[1]{\textcolor[rgb]{0.31,0.60,0.02}{{#1}}}
\newcommand{\ImportTok}[1]{{#1}}
\newcommand{\CommentTok}[1]{\textcolor[rgb]{0.56,0.35,0.01}{\textit{{#1}}}}
\newcommand{\DocumentationTok}[1]{\textcolor[rgb]{0.56,0.35,0.01}{\textbf{\textit{{#1}}}}}
\newcommand{\AnnotationTok}[1]{\textcolor[rgb]{0.56,0.35,0.01}{\textbf{\textit{{#1}}}}}
\newcommand{\CommentVarTok}[1]{\textcolor[rgb]{0.56,0.35,0.01}{\textbf{\textit{{#1}}}}}
\newcommand{\OtherTok}[1]{\textcolor[rgb]{0.56,0.35,0.01}{{#1}}}
\newcommand{\FunctionTok}[1]{\textcolor[rgb]{0.00,0.00,0.00}{{#1}}}
\newcommand{\VariableTok}[1]{\textcolor[rgb]{0.00,0.00,0.00}{{#1}}}
\newcommand{\ControlFlowTok}[1]{\textcolor[rgb]{0.13,0.29,0.53}{\textbf{{#1}}}}
\newcommand{\OperatorTok}[1]{\textcolor[rgb]{0.81,0.36,0.00}{\textbf{{#1}}}}
\newcommand{\BuiltInTok}[1]{{#1}}
\newcommand{\ExtensionTok}[1]{{#1}}
\newcommand{\PreprocessorTok}[1]{\textcolor[rgb]{0.56,0.35,0.01}{\textit{{#1}}}}
\newcommand{\AttributeTok}[1]{\textcolor[rgb]{0.77,0.63,0.00}{{#1}}}
\newcommand{\RegionMarkerTok}[1]{{#1}}
\newcommand{\InformationTok}[1]{\textcolor[rgb]{0.56,0.35,0.01}{\textbf{\textit{{#1}}}}}
\newcommand{\WarningTok}[1]{\textcolor[rgb]{0.56,0.35,0.01}{\textbf{\textit{{#1}}}}}
\newcommand{\AlertTok}[1]{\textcolor[rgb]{0.94,0.16,0.16}{{#1}}}
\newcommand{\ErrorTok}[1]{\textcolor[rgb]{0.64,0.00,0.00}{\textbf{{#1}}}}
\newcommand{\NormalTok}[1]{{#1}}
\usepackage{graphicx,grffile}
\makeatletter
\def\maxwidth{\ifdim\Gin@nat@width>\linewidth\linewidth\else\Gin@nat@width\fi}
\def\maxheight{\ifdim\Gin@nat@height>\textheight\textheight\else\Gin@nat@height\fi}
\makeatother
% Scale images if necessary, so that they will not overflow the page
% margins by default, and it is still possible to overwrite the defaults
% using explicit options in \includegraphics[width, height, ...]{}
\setkeys{Gin}{width=\maxwidth,height=\maxheight,keepaspectratio}
\setlength{\parindent}{0pt}
\setlength{\parskip}{6pt plus 2pt minus 1pt}
\setlength{\emergencystretch}{3em}  % prevent overfull lines
\providecommand{\tightlist}{%
  \setlength{\itemsep}{0pt}\setlength{\parskip}{0pt}}
\setcounter{secnumdepth}{5}

%%% Use protect on footnotes to avoid problems with footnotes in titles
\let\rmarkdownfootnote\footnote%
\def\footnote{\protect\rmarkdownfootnote}

%%% Change title format to be more compact
\usepackage{titling}

% Create subtitle command for use in maketitle
\newcommand{\subtitle}[1]{
  \posttitle{
    \begin{center}\large#1\end{center}
    }
}

\setlength{\droptitle}{-2em}
  \title{An introduction to knitcitations}
  \pretitle{\vspace{\droptitle}\centering\huge}
  \posttitle{\par}
  \author{Carl Boettiger}
  \preauthor{\centering\large\emph}
  \postauthor{\par}
  \predate{\centering\large\emph}
  \postdate{\par}
  \date{27 May 2014}


% Redefines (sub)paragraphs to behave more like sections
\ifx\paragraph\undefined\else
\let\oldparagraph\paragraph
\renewcommand{\paragraph}[1]{\oldparagraph{#1}\mbox{}}
\fi
\ifx\subparagraph\undefined\else
\let\oldsubparagraph\subparagraph
\renewcommand{\subparagraph}[1]{\oldsubparagraph{#1}\mbox{}}
\fi

\begin{document}
\maketitle

{
\hypersetup{linkcolor=black}
\setcounter{tocdepth}{2}
\tableofcontents
}
\section{knitcitations}\label{knitcitations}

\begin{itemize}
\tightlist
\item
  \textbf{Author}: \href{http://www.carlboettiger.info/}{Carl Boettiger}
\item
  \textbf{License}: \href{http://opensource.org/licenses/MIT}{MIT}
\item
  \href{https://github.com/cboettig/knitcitations}{Package source code
  on Github}
\item
  \href{https://github.com/cboettig/knitcitations/issues}{\textbf{Submit
  Bugs and feature requests}}
\end{itemize}

\texttt{knitcitations} is an R package designed to add dynamic citations
to dynamic documents created with
\href{https://github.com/yihui/knitr}{Yihui's knitr package}.

\subsection{Installation}\label{installation}

Install the development version directly from Github

\begin{Shaded}
\begin{Highlighting}[]
\KeywordTok{library}\NormalTok{(devtools)}
\KeywordTok{install_github}\NormalTok{(}\StringTok{"cboettig/knitcitations"}\NormalTok{)}
\end{Highlighting}
\end{Shaded}

Or install the current release from your CRAN mirror with
\texttt{install.packages("knitcitations")}.

\subsection{Quick start: rmarkdown (pandoc)
mode}\label{quick-start-rmarkdown-pandoc-mode}

Start by loading the library. It is usually good to also clear the
bibliographic environment after loading the library, in case any
citations are already stored there:

\begin{Shaded}
\begin{Highlighting}[]
\KeywordTok{setwd}\NormalTok{(}\StringTok{"D:/Google Drive/PHD/Progress/phdprogress1"}\NormalTok{)}
\KeywordTok{library}\NormalTok{(}\StringTok{"knitcitations"}\NormalTok{)}
\KeywordTok{cleanbib}\NormalTok{()}
\end{Highlighting}
\end{Shaded}

Set pandoc as the default format:

\begin{Shaded}
\begin{Highlighting}[]
\KeywordTok{options}\NormalTok{(}\StringTok{"citation_format"} \NormalTok{=}\StringTok{ "pandoc"}\NormalTok{)}
\end{Highlighting}
\end{Shaded}

(Note: The old method will eventually be depricated. For documents using
\texttt{knitcitations\ \textless{}=\ 0.5} it will become necessary to
set this as \texttt{"compatibility"}).

\subsubsection{Cite by DOI}\label{cite-by-doi}

Cite an article by DOI and the full citation information is gathered
automatically. By default this now generates a citation in
pandoc-flavored-markdown format. We use the inline command
\texttt{citep("10.1890/11-0011.1")} to create this citation
\citep{Abrams_2012}.

An in-text citation is generated with \texttt{citet}, such as
\texttt{citet("10.1098/rspb.2013.1372")} creating the citation to
\citet{Boettiger_2013}.

\subsubsection{Cite by URL}\label{cite-by-url}

Not all the literature we may wish to cite includes DOIs, such as
\href{http://arxiv.org}{arXiv} preprints, Wikipedia pages, or other
academic blogs. Even when a DOI is present it is not always trivial to
locate. With version 0.4-0, \texttt{knitcitations} can produce citations
given any URL using the
\href{http://greycite.knowledgeblog.org}{Greycite API}. For instance, we
can use the call \texttt{citep("http://knowledgeblog.org/greycite")} to
generate the citation to the Greycite tool \citep{greycite32194}.

\subsubsection{Cite bibtex and bibentry objects
directly}\label{cite-bibtex-and-bibentry-objects-directly}

We can also use \texttt{bibentry} objects such as R provides for citing
packages (using R's \texttt{citation()} function):
\texttt{citep(citation("knitr")} produces
\citep{Xie_2015, Xie_2015a, Xie_2014}. Note that this package includes
citations to three objects, and pandoc correctly avoids duplicating the
author names. In pandoc mode, we can still use traditional
pandoc-markdown citations like \texttt{@Boettiger\_2013} which will
render as \citet{Boettiger_2013} without any R code, provided the
citation is already in the \texttt{.bib} file we name (see below).

\subsubsection{Re-using Keys}\label{re-using-keys}

When the citation is called, a key in the format
\texttt{FirstAuthorsLastName\_Year} is automatically created for this
citation, so we can now continue to cite this article without
remembering the DOI, using the command \texttt{citep("Abrams\_2012")}
creates the citation \citep{Abrams_2012} without mistaking it for a new
article.

\subsubsection{Displaying the final
bibliography}\label{displaying-the-final-bibliography}

At the end of the document, include a chunk containing the command:

\begin{Shaded}
\begin{Highlighting}[]
\KeywordTok{write.bibtex}\NormalTok{(}\DataTypeTok{file=}\StringTok{"references.bib"}\NormalTok{)}
\end{Highlighting}
\end{Shaded}

Use the chunk options \texttt{echo=FALSE} and \texttt{message=FALSE} to
hide the chunk command and output.

This creates a Bibtex file with the name given.
\href{http://johnmacfarlane.net/pandoc}{Pandoc} can then be used to
compile the markdown into HTML, MS Word, LaTeX, PDF, or many other
formats, each with the desired journal styling. Pandoc is now integrated
with \href{http://rstudio.com}{RStudio} through the
\href{http://rmarkdown.rstudio.com}{rmarkdown} package. Pandoc appends
these references to the end of the markdown document automatically. In
this example, we have added a yaml header to our Rmd file which
indicates the name of the bib file being used, and the optional link to
a \href{https://github.com/citation-style-language/styles}{CSL}
stylesheet which formats the output for the ESA journals:

\begin{Shaded}
\begin{Highlighting}[]
\OtherTok{---}
\FunctionTok{bibliography:} \StringTok{"references.bib"}
\FunctionTok{csl:} \StringTok{"ecology.csl"}
\FunctionTok{output:}
  \NormalTok{html_document}
\OtherTok{---}
\end{Highlighting}
\end{Shaded}

\section{Example file for RStudio /
rmarkdown}\label{example-file-for-rstudio-rmarkdown}

This vignette itself is written as an \texttt{.Rmd} file with the
\texttt{yaml} header discussed above for working with RStudio's
\texttt{knit} buttons or the rmarkdown R package. You can see the
\href{https://raw.githubusercontent.com/cboettig/knitcitations/master/vignettes/tutorial.Rmd}{tutorial
source file here}. Calling \texttt{rmarkdown::render("tutorial.Rmd")}
from R on the tutorial compiles the output markdown, with references in
the format of the ESA journals.

\renewcommand\refname{References}
\bibliography{references.bib}

\end{document}
